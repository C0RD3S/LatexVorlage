% ------------------------------------------------------------------------
% Anpassung des Seitenlayouts  siehe Seitenstil.tex
% --------------------------------------------------------------------------

\usepackage[headsepline,automark]{scrlayer-scrpage}

\usepackage{listingsutf8}
\lstset{language=SQL, 
	commentstyle=\color{green}, backgroundcolor=\color{white}, 
	keywordstyle=\color{blue},
	basicstyle = \ttfamily \color{black} \footnotesize, 
	numbers=left, 
	stepnumber=1, 
	numbersep=10pt, 
	tabsize=4, 
	captionpos=b,
	breaklines=true,
	postbreak=\mbox{\textcolor{red}{$\hookrightarrow$}\space},
}

\renewcommand{\lstlistingname}{Code} % Damit Codebeispiele mit Code betitelt werden.


% ------------------------------------------------------------------------
% Anpassung an Landessprache 
% Verwendet globale Option german siehe \documentclass
% ------------------------------------------------------------------------

\usepackage[ngerman]{babel}
\usepackage{pifont}
\usepackage[utf8]{inputenc}
\usepackage{amsmath,amssymb,units, gensymb}
\usepackage{eurosym} 
\usepackage{wrapfig,caption, placeins}
\captionsetup{
	format = plain,
	justification = centering,
	labelsep = newline,
	singlelinecheck = false,
	labelfont = bf,
	font = small
}
\usepackage{mathptmx,charter,helvet,courier}
\usepackage[printonlyused, smaller, nolist]{acronym} %Entferne nolist aus den Argumenten, damit ein Verzeichnis der Abkürzungen erstellt wird. 
\usepackage{wrapfig}
\usepackage{tabularx}
\usepackage{booktabs}
\usepackage{tablefootnote}

% -------------------------------------------------------------------------
% Umlaute
% Umlaute/Sonderzeichen wie äöüß direkt im Quelltext verwenden (CodePage).
% Erlaubt automatische Trennung von Worten mit Umlauten.
% -------------------------------------------------------------------------

\usepackage[utf8]{inputenc}
\usepackage[T1]{fontenc}
\usepackage{ae}			 	% "schöneres" ä
\usepackage{textcomp} 		% Euro-Zeichen etc.
\usepackage[german=quotes]{csquotes}

% -------------------------------------------------------------------------
% Verwendung von Blindtext zur Layout gestaltung
% -------------------------------------------------------------------------
\usepackage{blindtext}

%Zur Verwendung von BibLaTex
\usepackage[backend=biber, 
            style=numeric,
            sorting=nty,
            abbreviate = true,
            pagetracker = true]{biblatex}


% -------------------------------------------------------------------------
% Grafiken 
% Einbinden von Grafiken [draft oder final]
% Option [draft] bindet Bilder nicht ein - auch globale Option
% -------------------------------------------------------------------------
\usepackage[final]{graphicx}
\graphicspath{{Grafiken/}} 	% Dort liegen die Bilder des Dokuments

% Befehle aus AMSTeX für mathematische Symbole z.B. \boldsymbol \mathbb ----
\usepackage{amsmath,amsfonts}

% Für Index-Ausgabe; \printindex -------------------------------------------
\usepackage{makeidx}

% Einfache Definition der Zeilenabstände und Seitenränder etc. -------------
\usepackage{setspace}
\usepackage{geometry}


%% Zum Umfließen von Bildern -------------------------------------------------
\usepackage{floatflt}


% Lange URLs umbrechen etc. -------------------------------------------------
\usepackage{url}

% für lange Tabellen
\usepackage{longtable}
\usepackage{multirow}
\usepackage{array}
\usepackage{ragged2e}
\usepackage{lscape}
\usepackage{colortbl} %farbige hinterlegung
%\usepackage{enumitem}

% Spaltendefinition mit definierter Breite ---------------------
\newcolumntype{L}[1]{>{\raggedright\hspace{0pt}}p{#1}} % linksbündig mit Breitenangabe
\newcolumntype{C}[1]{>{\centering\arraybackslash}p{#1}} % zentriert mit Breitenangabe
\newcolumntype{R}[1]{>{\raggedleft\arraybackslash}p{#1}} % rechtsbündig mit Breitenangabe
\newcolumntype{w}[1]{>{\raggedleft\hspace{0pt}}p{#1}}


% Formatierung von Listen ändern
\usepackage{paralist}
% Standardeinstellungen:
\setdefaultleftmargin{2.5em}{2.2em}{1.87em}{1.7em}{1em}{1em}


% Zur Richtigen Verwendung von Einheiten
\usepackage[locale=DE]{siunitx}

\sisetup{
	mode = text,
	detect-family,			%Formatierungserkennung der Umgebung
	detect-weight,			%Formatierungserkennung der Umgebung  
	exponent-product = \cdot,
	product-units = single,
	number-unit-separator={~},
	output-decimal-marker={\text{,}},
    separate-uncertainty, 			%Angabe bei Fehlern(5 +- 3)
	%math-rm=\mathsf,
	%text-rm=\sffamily,
	range-phrase = { - }			%Bei Áusgabe eines Bereichs 5-7V
}
\DeclareSIUnit{\var}{var}
\DeclareSIUnit{\mio}{Mio.\,}
%--Mögliche Befehle für siunitx----------------------------------------------
%\num[Optionen]{Zahl}  
%\si[Optionen]{Einheit}  
%\SI[Optionen]{Zahl}[per-Einheit]{Einheit}  
%\numlist[Optionen]{Zahl;Zahl;Zahl}  
%\numrage[Optionen]{Zahl Anfang}{Zahl Ende}  
%\SIlist[Optionen]{Zahlen}{Einheit}  
%\SIrange[Optionen]]{Zahl Anfang}{Zahl Ende}{Einheit}  
%\tablenum[Optionen]{Zahl}  
%\ang[Optionen]{Winkel} 
% Option sollte nicht genutzt werden!
%-----------------------------------------------------------------------------

\usepackage{xcolor}
\usepackage{caption}
\usepackage{float}
\usepackage{subfig}
\usepackage{pdfpages}

% pdf-Optionen  --------------------------------------------------------------
\usepackage[
bookmarks,
bookmarksopen=true,
pdftitle={\titel},
pdfauthor={\autorC},
pdfcreator={\autorC},
pdfsubject={\titel},
pdfkeywords={\titel},
colorlinks=true,
%linkcolor=red, % einfache interne Verknüpfungen
%anchorcolor=black,% Ankertext
%citecolor=blue, % Verweise auf Literaturverzeichniseinträge im Text
%filecolor=magenta, % Verkn�pfungen, die lokale Dateien �ffnen
%menucolor=red, % Acrobat-Men�punkte
%urlcolor=cyan, 
% f�r die Druckversion k�nnen die Farben ausgeschaltet werden:
linkcolor=black, % einfache interne Verkn�pfungen
anchorcolor=black,% Ankertext
citecolor=black, % Verweise auf Literaturverzeichniseinträge im Text
filecolor=black, % Verkn�pfungen, die lokale Dateien �ffnen
menucolor=black, % Acrobat-Men�punkte
urlcolor=black, 
%backref,
%pagebackref,
plainpages=false,% zur korrekten Erstellung der Bookmarks
pdfpagelabels,% zur korrekten Erstellung der Bookmarks
hypertexnames=false,% zur korrekten Erstellung der Bookmarks
linktoc=all%Sowohl Seitenzahl als auch Text als Link
]{hyperref}